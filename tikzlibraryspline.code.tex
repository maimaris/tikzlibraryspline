%% tikzlibraryspline.code.tex
%% Copyright (C) 2018 Stephen Checkoway
%
% This work may be distributed and/or modified under the
% conditions of the LaTeX Project Public License version 1.3c.
%
% Version 1.3c of this license is available at
%   http://www.latex-project.org/lppl/lppl-1-3c/
% and is included in the LICENSE file.
%
% This work has the LPPL maintenance status `maintained'.
%
% The Current Maintainer of this work is Stephen Checkoway.
%
% This work consists of the file tikzlibraryspline.code.tex.

%% Some notes about the implementation.
% This library computes cubic bezier splines through some given points K[0],
% K[1], ..., K[n]. In the case of a closed curve, the user specifies only
% K[0], K[1], ..., K[n-1] and K[n] is taken to be K[0].
%
% A spline (for our purposes) is a piecewise cubic curve consisting of n
% cubic curves connected where segment i starts at point K[i] and ends at
% point K[i+1].
%
% This code and documentation uses the term "knot" to refer to the K[i].
% Traditionally, a knot refers to "time" values, one of the t[0], t[1], ...,
% t[n] values such that S[t[i]] = K[i]. Hopefully, this won't cause too much
% confusion.
%
% The bezier curve control points are calculated by writing the parametric
% equations of each cubic curve and constraining the first and second
% derivatives.
%
% Specifically, we have the n functions
%
%    f_i(t) = (1-t)^3 K[i] + (1-t)^2 t P[i] + (1-t)t^2 Q[i] + t^3 K[i+1]
%
% for i in {0,1,...,n-1} and 0 <= t <= 1. For a closed spline, K[n] = K[0].
%
% For a closed curve, we construct 2n equations in 2n variables by
% requiring differentiating and setting f_{i+1}'(0) = f_i'(1) and
% f_{i+1}''(0) = f_i''(1).
%
% Otherwise, we have 2n-2 equations of that form for the interior knot points.
% By setting f_0''(0) = 0 and f_{n-1}''(1) = 0, we get a total of 2n
% functions.
%
% By some simple algebraic substitution, we can make a system of n equations
% in P[0], P[1], ..., P[n-1]. In the closed case, we have
%
%     P[i-1] + 4P[i] + P[i+1] = 4K[i] + 2K[i+1].
%
% where all indices are taken modulo n. This leads to the linear system AP = R
% where R is the vector of values from the RHS and
%
%    A = [4 1       1
%         1 4 1
%           1 4 1
%               ...
%         1       1 4]
%
% We can solve this linear system in linear time by using the Sherman-Morrison
% formula. In essence, we compute A'P = R, A'z = u where
%
%    A' = [3 1
%          1 4 1
%            1 4 1
%                ...
%                  1 3], and
%     u = [1 0 ... 0 1]^T.
%
% This choice of u is unusual but works well with the fixed point numbers
% we're working with.
%
% Finally, if x[i] and y[i] are the coordinates of P[i], we compute
%
%    fx = (x[0] + x[n-1])/(1 + z[0] + z[n-1]), and
%    fy = (y[0] + y[n-1])/(1 + z[0] + z[n-1]).
%
% and finally update P[i] by setting
%
%    x[i] <- x[i] - fx*z[i]
%    y[i] <- y[i] - fy*z[i].
%
% This gives us the first control point for each segment. From the curve
% equations, we get
%
%    Q[i] = 2K[i+1] - P[i+1]
%
% where again, the indices are taken modulo n.
%
% The situation for the non-closed case is simpler. We have the system of n
% equations in P[0], P[1], ..., P[n-1].
%
%               2P[0]    + P[1]   =  K[0]   + 2K[1]
%      P[i-1] + 4P[i]    + P[i+1] = 2K[i]   + 2K[i+1]  for 0 < i < n-1
%     2P[n-2] + 7P[n-1]           = 8K[n-1] + K[n].
%
% This leads to the linear equation BP = R where elements of R are again the
% RHS of the equations and
%
%    B = [2 1
%         1 4 1
%           1 4 1
%               ...
%                 2 7].
%
% Solving for P gives us the first control point of each segment. The second
% control point Q[i] can be computed via
%
%    Q[i]   = 2K[i+1] - P[i+1] for 0 <= i < n-1
%    Q[n-1] = (K[n] + P[n-1])/2.
%
% That just leaves us with the task of solving these linear systems. Note that
% both A' and B are diagonally dominant tridiagonal matrices. We can solve
% these using the Thomas algorithm.
%
% In general, a tridiagonal matrix has the form
%    [b[0] c[0]
%     a[1] b[1] c[1]
%          a[2] b[2] c[2]
%                       ...
%                         a[n-1] b[n-1]].
% In our case,
%
%    a[i]   = 1 for 0 <= i < n-1
%    a[n-1] = 1 if the spline is closed, otherwise 2
%    b[0]   = 3 if the spline is closed, otherwise 2
%    b[i]   = 4 for 1 <= i < n-1
%    b[n-1] = 3 if the spline is closed, otherwise 7
%    c[i]   = 1 for 0 <= i <= n-1
%
% There are several variants of the algorithm. This performs the following
% steps.
%
% First, the elements below the main diagonal are eliminated by iterating over
% each row i > 0 and replacing it with an appropriate linear combination with
% the previous row. That is, for i=1 up to i=n-1,
%
%    b[i] <- b[i] - a[i]c[i-1]/b[i-1]
%    R[i] <- R[i] - a[i]R[i-1]/b[i-1]
%    u[i] <- u[i] - a[i]u[i-1]/b[i-1] for the closed spline case.
%
% Note that this can be simplified since iterations i=1 through i=n-2
% all have a[i] = 1, b[i] = 4, c[i] = 1, and u[i] = 0.
%
% In fact, the modified b[i] is a function of b[0] (which is either 3 or 2)
% and can be computed via the recurrance b[i] = 4 - 1/b[i]. This converges to
% 2+sqrt(3) regardless of b[0]. For b[0] = 2 or 3, and i > 6, b[i] - b[6] is
% smaller than we can represent. Thus, we can just have a look up table for
% the first 7 values for each of the two starting conditions.
%
% once the lower diagonal has been eliminated, back substitution gives our
% solutions. That is,
%
%    P[n-1] = R[n-1]/b[n-1]
%    z[n-1] = u[n-1]/b[n-1]
%
% and for i=n-2 down to i=0,
%
%    P[i] = (R[i] - c[i]P[i+1])/b[i]
%    z[i] = (u[i] - c[i]z[i+1])/b[i].
%
% Again, we have c[i] = 1 and can look b[i] up in a table.
%
% That's the whole algorithm.
%
% Once we have the control points P[i] and Q[i], we can use pgf to add the
% cublic curve which in pseudo-TikZ notation is:
%
%    .. controls P[i] and Q[i] .. K[i+1]
%
% If nodes are used as knot points, the intersections library is used to
% compute the intersection point between the shape's background path and the
% curve and pgf is used to add just the segment of the curve that's visible.
% TikZ's normal behavior for (A) .. controls (B) and (C) .. (D) when one of
% (A) or (D) is s node is not great, but it's okay. For the spline, it looks
% terrible.

\ProvidesFile{tikzlibraryspline.code.tex}%
  [2018/03/13 v1.0.0 Cubic bezier spline library]
\usepgflibrary{intersections}

\newif\iftikz@spline@closed
\newif\iftikz@spline@usetime
\newcount\tikz@spline@i

\let\tikz@spline@debug=\pgfutil@gobble
\let\tikz@spline@name=\relax

\tikzset{
  spline through/.style={%
    to path={%
      \pgfextra
      \tikz@spline@closedfalse
      \tikz@splinethrough{(\tikztostart)#1(\tikztotarget)}%
      \tikztonodes
    },%
  },
  spline through/.value required,
  closed spline through/.style={%
    to path={%
      \pgfextra
      \tikz@spline@closedtrue
      \let\tikztotarget=\tikztostart
      \tikz@splinethrough{(\tikztostart)#1}%
      \tikztonodes
    }%
  },%
  closed spline through/.value required,
  spline coordinates/.store in=\tikz@spline@name,
  spline coordinates/.default=spline,
  spline debug/.store in=\tikz@spline@debug,
  spline debug/.default=\typeout
}

% Helpers
\def\tikz@spline@save@C#1{%
  \expandafter\edef\csname tikz@spline@C\number#1\endcsname
}

\def\tikz@spline@get@C#1{%
  \csname tikz@spline@C\number#1\endcsname
}

\def\tikz@spline@save@K#1{%
  \expandafter\edef\csname tikz@spline@K\number#1\endcsname
}

\def\tikz@spline@get@K#1{%
  \csname tikz@spline@K\number#1\endcsname
}

% Start the spline construction process.
\def\tikz@splinethrough#1{%
  % Step 1. Parse all of the knots.
  \tikz@spline@i=0
  \tikz@scan@one@point\tikz@spline@knot#1\pgf@stop
}

% Handle a knot by saving the coordinates and the name.
\def\tikz@spline@knot#1{%
  \tikz@make@last@position{#1}%
  \tikz@spline@debug{Parsed point: (\the\tikz@lastx,\the\tikz@lasty) \detokenize{#1}}%
  \tikz@spline@save@K{\tikz@spline@i}{%
    \pgf@x=\the\tikz@lastx
    \pgf@y=\the\tikz@lasty
    \iftikz@shapeborder
      \def\noexpand\tikz@shapeborder@name{\tikz@shapeborder@name}%
    \else
      \let\noexpand\tikz@shapeborder@name\relax
    \fi
  }%
  \advance\tikz@spline@i by1
  \pgfutil@ifnextchar\pgf@stop{%
    \tikz@spline@check@knots
  }{%
    \tikz@scan@one@point\tikz@spline@knot
  }%
}

% Check if we can proceed with the spline contruction.
\def\tikz@spline@check@knots\pgf@stop{%
  % Step 2. Make sure that a closed spline has at least three knots. If the
  % spline is not closed, then we have one fewer segments than knots.
  \let\tikz@next=\tikz@splinethroughA
  \iftikz@spline@closed
    \ifnum\tikz@spline@i<3
      \tikzerror{Closed splines require at least three points}%
      \edef\tikz@next{%
        \noexpand\endpgfextra
        --(\the\tikz@lastx,\the\tikz@lasty)%
      }%
    \fi
  \else
    % If the curve is not closed, there are n-1 segments with n knots
    \advance\tikz@spline@i by-1
  \fi
  % Keep track of how many bezier curve segments we'll be creating.
  \tikz@spline@debug{Creating a spline with \the\tikz@spline@i\space segments}%
  \mathchardef\tikz@spline@segments=\tikz@spline@i
  \tikz@next
}

% Continue the spline construction.
\def\tikz@splinethroughA{%
  % Step 3. Compute the control points P[i].
  \tikz@spline@computeP
  % Step 4. Compute the control points Q[i].
  \tikz@spline@computeQ
  % Step 5. Add the bezier curve segments to the path.
  \tikz@spline@path
  % Step 6. Optionally add coordinates for the control and knot points.
  \ifx\tikz@spline@name\relax\else
    \tikz@spline@i=0
    \tikz@spline@coordinates
    \xdef\tikzsplinesegments{\number\tikz@spline@segments}%
  \fi
  % Done!
  \endpgfextra
}

% Look up the new b (which we're calling bb) values in our table. Read the
% description of the algorithm for how these are computed. For i=n-1, we look
% up n-2 and then compute bb[n-1] = b[n-1] - a[n-1]/bb[n-2].
\def\tikz@spline@lookup@bb#1{%
  \c@pgf@counta=#1\relax
  \advance\c@pgf@counta by1
  \ifnum\c@pgf@counta=\tikz@spline@segments
    % bb[n-1]
    \advance\c@pgf@counta by-2
    \iftikz@spline@closed
      % a/b = bb[n-2]
      \tikz@spline@lookup@closed{\c@pgf@counta}%
      % We want bb[n-1] = (3a - b)/a so set d = 3a - b.
      \c@pgf@countd=\c@pgf@counta
      \multiply\c@pgf@countd by3
      \advance\c@pgf@countd by-\c@pgf@countb
    \else
      % a/b = bb[n-2]
      \tikz@spline@lookup@open{\c@pgf@counta}%
      % We want bb[n-1] = (7a - 2b)/a so set d = 7a - 2b.
      \c@pgf@countd=\c@pgf@counta
      \multiply\c@pgf@countd by7
      \advance\c@pgf@countd by-\c@pgf@countb
      \advance\c@pgf@countd by-\c@pgf@countb
    \fi
    \c@pgf@countb=\c@pgf@counta
    \c@pgf@counta=\c@pgf@countd
  \else
    % bb[i]
    \advance\c@pgf@counta by-1
    \iftikz@spline@closed
      \tikz@spline@lookup@closed{\c@pgf@counta}%
    \else
      \tikz@spline@lookup@open{\c@pgf@counta}%
    \fi
  \fi
}

\def\tikz@spline@lookup@closed#1{%
  % First, assign to a and then assign to b
  \afterassignment\c@pgf@countb
  \c@pgf@counta=\ifcase#1%
        3 1%
    \or 11 3%
    \or 41 11%
    \or 153 41%
    \or 571 153%
    \or 2131 571%
    \else 7953 2131%
  \fi
  \relax % We need a single token to end the second expansion.
}

\def\tikz@spline@lookup@open#1{%
  \afterassignment\c@pgf@countb
  \c@pgf@counta=\ifcase#1%
        2 1%
    \or 7 2%
    \or 26 7%
    \or 97 26%
    \or 362 97%
    \or 1351 362%
    \else 5042 1351%
  \fi
  \relax
}

% Step 3. Compute the control points P[i] by solving AP = R.
\def\tikz@spline@computeP{%
  % Perform the first step of the forward sweep of solving the tridiagonal (or
  % cyclic) matrix equation.
  \tikz@spline@get@K{0}%
  \iftikz@spline@closed
    % R[0] = 4K[0] + 2K[1]
    \pgf@xa=4\pgf@x
    \pgf@ya=4\pgf@y
  \else
    % R[0] = K[0] + 2*K[1]
    \pgf@xa=\pgf@x
    \pgf@ya=\pgf@y
  \fi
  \tikz@spline@get@K{1}%
  \advance\pgf@xa by2\pgf@x
  \advance\pgf@ya by2\pgf@y
  % x[0] = rx[0]
  % y[0] = ry[0]
  % z[0] = 1
  \pgf@xc=1pt
  \tikz@spline@save@C{0}{%
    \pgf@xa=\the\pgf@xa
    \pgf@ya=\the\pgf@ya
    \pgf@xc=\the\pgf@xc
  }%
  \tikz@spline@lookup@bb{0}%
  \tikz@spline@debug{%
    0: bb=\the\c@pgf@counta/\the\c@pgf@countb,
    x=\the\pgf@xa,
    y=\the\pgf@ya,
    z=\the\pgf@xc%
  }%
  \tikz@spline@i=1
  \c@pgf@countc=2
  \tikz@spline@computePA
}

% Step 3a. Handle all forward sweeping steps up to segments-2.
\def\tikz@spline@computePA{%
  % At this point,
  % \pgf@x/\pgf@y = K[i]
  % \pgf@xa/\pgf@ya = P[i-1]
  % \pgf@xc = z[i-1]
  % \c@pgf@counta/\c@pgf@countb = bb[i-1]
  %
  % R[i] = 4*K[i] + 2*K[i+1]
  \pgf@xb=4\pgf@x
  \pgf@yb=4\pgf@y
  \tikz@spline@get@K{\c@pgf@countc}%
  \advance\pgf@xb by2\pgf@x
  \advance\pgf@yb by2\pgf@y
  % P[i] = R[i] - P[i-1]/bb[i-1]
  % z[i] = 0 - z[i-1]/bb[i-1]
  \pgf@xa=\dimexpr\pgf@xb-\pgf@xa*\c@pgf@countb/\c@pgf@counta\relax
  \pgf@ya=\dimexpr\pgf@yb-\pgf@ya*\c@pgf@countb/\c@pgf@counta\relax
  \pgf@xc=\dimexpr-\pgf@xc*\c@pgf@countb/\c@pgf@counta\relax
  \tikz@spline@save@C{\tikz@spline@i}{%
    \pgf@xa=\the\pgf@xa
    \pgf@ya=\the\pgf@ya
    \pgf@xc=\the\pgf@xc
  }%
  \tikz@spline@lookup@bb{\tikz@spline@i}%
  \tikz@spline@debug{%
    \the\tikz@spline@i:
    bb=\the\c@pgf@counta/\the\c@pgf@countb,
    x=\the\pgf@xa,
    y=\the\pgf@ya,
    z=\the\pgf@xc%
  }%
  \advance\tikz@spline@i by1
  \advance\c@pgf@countc by1
  \ifnum\c@pgf@countc=\tikz@spline@segments
    \expandafter\tikz@spline@computePB
  \else
    \expandafter\tikz@spline@computePA
  \fi
}


% Step 3b. Handle the last iteration of the forward sweep and the first
% iteration of the backwards substitution.
\def\tikz@spline@computePB{%
  % At this point,
  % \pgf@x/\pgf@y = K[n-1]
  % \pgf@xa/\pgf@ya = P[n-2]
  % \pgf@xc = z[n-2]
  % \c@pgf@counta/\c@pgf@countb = bb[n-2]
  \iftikz@spline@closed
    \pgf@xb=4\pgf@x
    \pgf@yb=4\pgf@y
    \tikz@spline@get@K{0}%
    \advance\pgf@xb by2\pgf@x
    \advance\pgf@yb by2\pgf@y
    % P[n-1] = R[n-1] - P[n-2]/bb[n-2]
    % z[n-1] = 1 - z[n-2]/bb[n-2]
  \else
    % R[n-1] = 8K[n-1] + K[n]
    \pgf@xb=8\pgf@x
    \pgf@yb=8\pgf@y
    \tikz@spline@get@K{\c@pgf@countc}%
    \advance\pgf@xb by\pgf@x
    \advance\pgf@yb by\pgf@y
    % P[n-1] = R[n-1] - 2P[n-2]/bb[n-2]
    \advance\c@pgf@countb by\c@pgf@countb
  \fi
  \pgf@xa=\dimexpr\pgf@xb-\pgf@xa*\c@pgf@countb/\c@pgf@counta\relax
  \pgf@ya=\dimexpr\pgf@yb-\pgf@ya*\c@pgf@countb/\c@pgf@counta\relax
  \pgf@xc=\dimexpr1pt-\pgf@xc*\c@pgf@countb/\c@pgf@counta\relax
  % No need to save these since we're going to use them immediately.
  %
  % P[n-1] = P[n-1]/bb[n-1]
  % z[n-1] = z[n-1]/bb[n-1]
  \tikz@spline@lookup@bb{\tikz@spline@i}%
  \tikz@spline@debug{%
    \the\tikz@spline@i:
    bb=\the\c@pgf@counta/\the\c@pgf@countb,
    x=\the\pgf@xa,
    y=\the\pgf@ya,
    z=\the\pgf@xc%
  }%
  % Store P[n-1] in \pgf@xb/\pgf@yb and z[n-1] in \pgf@yc
  \pgf@xb=\dimexpr\pgf@xa*\c@pgf@countb/\c@pgf@counta\relax
  \pgf@yb=\dimexpr\pgf@ya*\c@pgf@countb/\c@pgf@counta\relax
  \pgf@yc=\dimexpr\pgf@xc*\c@pgf@countb/\c@pgf@counta\relax
  \tikz@spline@save@C{\tikz@spline@i}{%
    \pgf@xa=\the\pgf@xb
    \pgf@ya=\the\pgf@yb
    \pgf@xc=\the\pgf@yc
  }%
  \tikz@spline@debug{%
    \the\tikz@spline@i:
    bb=\the\c@pgf@counta/\the\c@pgf@countb,
    x=\the\pgf@xb,
    y=\the\pgf@yb,
    z=\the\pgf@yc%
  }%
  \def\tikz@next{\tikz@spline@computePC}%
  \tikz@spline@computePC
}

% Step 3c. Handle all of the rest of the backwards substitutions.
\def\tikz@spline@computePC{%
  \advance\tikz@spline@i by-1
  % At this point,
  % \pgf@xb/\pgf@yb = P[i+1]
  % \pgf@yc = z[i+1]
  %
  % P[i] = (P[i] - P[i+1])/bb[i]
  % z[i] = (z[i] - z[i+1])/bb[i]
  \tikz@spline@lookup@bb{\tikz@spline@i}%
  \tikz@spline@get@C{\tikz@spline@i}%
  \pgf@xb=\dimexpr(\pgf@xa-\pgf@xb)*\c@pgf@countb/\c@pgf@counta\relax
  \pgf@yb=\dimexpr(\pgf@ya-\pgf@yb)*\c@pgf@countb/\c@pgf@counta\relax
  \pgf@yc=\dimexpr(\pgf@xc-\pgf@yc)*\c@pgf@countb/\c@pgf@counta\relax
  \tikz@spline@save@C{\tikz@spline@i}{%
    \pgf@xa=\the\pgf@xb
    \pgf@ya=\the\pgf@yb
    \pgf@xc=\the\pgf@yc
  }%
  \tikz@spline@debug{%
    \the\tikz@spline@i:
    bb=\the\c@pgf@counta/\the\c@pgf@countb,
    x=\the\pgf@xb,
    y=\the\pgf@yb,
    z=\the\pgf@yc%
  }%
  \ifnum\tikz@spline@i=0
    \tikz@spline@i=\tikz@spline@segments
    \iftikz@spline@closed
      % Goto step 3d.
      \def\tikz@next{\tikz@spline@computePD}%
    \else
      % Done with step 3.
      \let\tikz@next=\relax
    \fi
  \fi
  \tikz@next
}

% Step 3d. For a closed spline, compute the appropriate factor and then
% multiply P[i] by the factor for each i.
\def\tikz@spline@computePD{%
  % d = 1 + z[0] + z[n-1]
  % fx = (x[0] + x[n-1])/d
  % fy = (y[0] + y[n-1])/d
  \advance\tikz@spline@i by-1
  \tikz@spline@get@C{\tikz@spline@i}%
  \pgf@xb=\pgf@xa
  \pgf@yb=\pgf@ya
  \pgf@yc=\pgf@xc
  \tikz@spline@get@C{0}%
  \advance\pgf@xc by\pgf@yc
  \advance\pgf@xc by1pt
  \advance\pgf@xa by\pgf@xb
  \advance\pgf@ya by\pgf@yb
  \c@pgf@counta=\pgf@xc
  \pgf@xc=\dimexpr\pgf@xa*65536/\c@pgf@counta\relax
  \pgf@yc=\dimexpr\pgf@ya*65536/\c@pgf@counta\relax
  \edef\tikz@spline@fx{\pgf@sys@tonumber\pgf@xc}%
  \edef\tikz@spline@fy{\pgf@sys@tonumber\pgf@yc}%
  \tikz@spline@i=\tikz@spline@segments
  \tikz@spline@computePE
}

% Step 3e. For a closed spline only. Iterate through each segment and
% modify P appropriately.
\def\tikz@spline@computePE{%
  \advance\tikz@spline@i by-1
  \tikz@spline@get@C{\tikz@spline@i}%
  % Px[i] = Px[i] - fx*z[i]
  % Py[i] = Py[i] - fy*z[i]
  \advance\pgf@xa by-\tikz@spline@fx\pgf@xc
  \advance\pgf@ya by-\tikz@spline@fy\pgf@xc
  \tikz@spline@save@C{\tikz@spline@i}{%
    \pgf@xa=\the\pgf@xa
    \pgf@ya=\the\pgf@ya
  }%
  \ifnum\tikz@spline@i=0
    \let\tikz@spline@fx=\@undefined
    \let\tikz@spline@fy=\@undefined
  \else
    \expandafter\tikz@spline@computePE
  \fi
}

% Step 4. Compute control points Q[i].
\def\tikz@spline@computeQ{%
  \tikz@spline@i=\tikz@spline@segments
  \iftikz@spline@closed
    % Q[n-1] = 2K[0] - P[0]
    \advance\tikz@spline@i by-1
    \tikz@spline@get@K{0}%
    \tikz@spline@get@C{0}%
    \pgf@xb=2\pgf@x
    \pgf@yb=2\pgf@y
    \advance\pgf@xb by-\pgf@xa
    \advance\pgf@yb by-\pgf@ya
    \tikz@spline@get@C{\tikz@spline@i}%
  \else
    % Q[n-1] = (K[n] + P[n-1])/2
    \tikz@spline@get@K{\tikz@spline@i}%
    \advance\tikz@spline@i by-1
    \tikz@spline@get@C{\tikz@spline@i}%
    \pgf@xb=\pgf@x
    \pgf@yb=\pgf@y
    \advance\pgf@xb by\pgf@xa
    \advance\pgf@yb by\pgf@ya
    \divide\pgf@xb by2
    \divide\pgf@yb by2
  \fi
  \tikz@spline@save@C{\tikz@spline@i}{%
    \pgf@xa=\the\pgf@xa
    \pgf@ya=\the\pgf@ya
    \pgf@xb=\the\pgf@xb
    \pgf@yb=\the\pgf@yb
  }%
  \tikz@spline@debug{%
    P[\the\tikz@spline@i]=(\the\pgf@xa,\the\pgf@ya)
    Q[\the\tikz@spline@i]=(\the\pgf@xb,\the\pgf@yb)%
  }%
  \tikz@spline@computeQA
}

% Step 4a. Compute the rest of the Q[i] = 2K[i+1] - P[i+1]
\def\tikz@spline@computeQA{%
  \tikz@spline@get@K{\tikz@spline@i}%
  \tikz@spline@get@C{\tikz@spline@i}%
  \pgf@xb=2\pgf@x
  \pgf@yb=2\pgf@y
  \advance\pgf@xb by-\pgf@xa
  \advance\pgf@yb by-\pgf@ya
  \advance\tikz@spline@i by-1
  \tikz@spline@get@C{\tikz@spline@i}%
  \tikz@spline@save@C{\tikz@spline@i}{%
    \pgf@xa=\the\pgf@xa
    \pgf@ya=\the\pgf@ya
    \pgf@xb=\the\pgf@xb
    \pgf@yb=\the\pgf@yb
  }%
  \tikz@spline@debug{%
    P[\the\tikz@spline@i]=(\the\pgf@xa,\the\pgf@ya)
    Q[\the\tikz@spline@i]=(\the\pgf@xb,\the\pgf@yb)%
  }%
  \ifnum\tikz@spline@i=0\else
    \expandafter\tikz@spline@computeQA
  \fi
}

% Step 5. Add the bezier curve segments to the path.
\def\tikz@spline@path{%
  \tikz@spline@i=0
  \tikz@spline@pathA
  \let\tikz@timer=\tikz@spline@timer
}

\def\tikz@spline@select@segment#1{%
  \c@pgf@counta=#1\relax
  \tikz@spline@get@K{\c@pgf@counta}%
  \tikz@spline@get@C{\c@pgf@counta}%
  \edef\tikz@spline@zeroth{\noexpand\pgfqpoint{\the\pgf@x}{\the\pgf@y}}%
  \let\tikz@spline@startname=\tikz@shapeborder@name
  \edef\tikz@spline@first{\noexpand\pgfqpoint{\the\pgf@xa}{\the\pgf@ya}}%
  \edef\tikz@spline@second{\noexpand\pgfqpoint{\the\pgf@xb}{\the\pgf@yb}}%
  \advance\c@pgf@counta by1
  \ifnum\c@pgf@counta=\tikz@spline@segments
    \iftikz@spline@closed
      % The end knot is the start knot.
      \c@pgf@counta=0
    \fi
  \fi
  \tikz@spline@get@K{\c@pgf@counta}%
  \edef\tikz@spline@third{\noexpand\pgfqpoint{\the\pgf@x}{\the\pgf@y}}%
  \let\tikz@spline@endname=\tikz@shapeborder@name
}

% Step 5a.
\def\tikz@spline@pathA{%
  \tikz@spline@select@segment{\tikz@spline@i}%
  % If the start or end coordinates are shapes, calculate the intersection
  % times.
  \tikz@spline@usetimefalse
  \def\tikz@spline@timea{0}%
  \def\tikz@spline@timeb{1}%
  \ifx\tikz@spline@startname\relax\else
    \pgfutil@ifundefined{pgf@sh@bg@\csname pgf@sh@ns@\tikz@spline@startname\endcsname}{}{%
      % Defined.
      \tikz@spline@border@time{\tikz@spline@startname}{\tikz@spline@timea}{1}%
      \tikz@spline@usetimetrue
    }%
  \fi
  % Check for an end shape
  \ifx\tikz@spline@endname\relax\else
    \pgfutil@ifundefined{pgf@sh@bg@\csname pgf@sh@ns@\tikz@spline@endname\endcsname}{}{%
      \tikz@spline@border@time{\tikz@spline@endname}{\tikz@spline@timeb}{\pgfintersectionsolutions}%
      \tikz@spline@usetimetrue
    }%
  \fi
  % If either end point was a shape, use \pgfpathcurvebetweentime to draw part
  % of the curve.
  \iftikz@spline@usetime
    \pgfpathcurvebetweentime{\tikz@spline@timea}{\tikz@spline@timeb}%
      {\tikz@spline@zeroth}{\tikz@spline@first}{\tikz@spline@second}{\tikz@spline@third}%
  \else
    \pgfpathcurveto{\tikz@spline@first}{\tikz@spline@second}{\tikz@spline@third}%
  \fi
  \ifnum\tikz@spline@i=0
    % Save the start time for the timer.
    \let\tikz@spline@starttime=\tikz@spline@timea
  \fi
  \advance\tikz@spline@i by1
  \ifnum\tikz@spline@i=\tikz@spline@segments
    \ifdim\tikz@spline@timeb pt<1pt
      % \pgfpointcurveattime puts points on the tangent in xa/ya and xb/yb.
      % XXX: This was computed from the intersection before, we could reuse
      % it.
      \pgfpointcurveattime{\tikz@spline@timeb}{\tikz@spline@zeroth}%
        {\tikz@spline@first}{\tikz@spline@second}{\tikz@spline@third}%
      \edef\tikz@tangent{\noexpand\pgfqpoint{\the\pgf@xa}{\the\pgf@yb}}%
      \tikz@make@last@position{}% \pgf@x/\pgf@y already contain the position.
    \else
      % The end point of the curve is the last position and the second control
      % point is on the tangent.
      \let\tikz@tangent=\tikz@spline@second
      \tikz@make@last@position{\tikz@spline@third}%
    \fi
    \let\tikz@spline@endtime=\tikz@spline@timeb
  \else
    \expandafter\tikz@spline@pathA
  \fi
}

% \tikz@spline@border@time{node}{\time}{which solution}
\def\tikz@spline@border@time#1#2#3{
  \begingroup
  \pgfintersectionofpaths{%
    \pgfpathmoveto{\tikz@spline@zeroth}%
    \pgfpathcurveto{\tikz@spline@first}{\tikz@spline@second}{\tikz@spline@third}%
  }{%
    % Install special macros (cargo-cult programming, I don't know what these
    % are or why they're important)
    \csname pgf@sh@ma@#1\endcsname
    % Install special coordinates (things like \radius and \centerpoint)
    \csname pgf@sh@np@#1\endcsname
    % Set the transform
    \pgfsettransform{\csname pgf@sh@nt@#1\endcsname}%
    % Create the background path
    \csname pgf@sh@bg@\csname pgf@sh@ns@#1\endcsname\endcsname
  }
  \ifnum\pgfintersectionsolutions=0
    \tikz@spline@debug{Unable to find an intersection between `#1' and the spline}
    \pgf@process{%
      \ifnum#3=0
        \tikz@spline@zeroth
      \else
        \tikz@spline@third
      \fi
    }%
  \else
    \pgfintersectiongetsolutiontimes{#3}{#2}{\tikz@spline@dummy}%
  \fi
  \edef\tikz@marshal{\endgroup\def\noexpand#2{#2}}%
  \tikz@marshal
}

% Transform \tikz@time to a point along the spline. If the start or end are
% shapes, then times 0 and 1 correspond to intersection points with the
% border.
\def\tikz@spline@timer{%
  % Compute the segment number in \tikz@spline@i
  \pgf@xa=\tikz@time \tikz@spline@segments
  \tikz@spline@i=\pgf@xa
  % Compute the fractional part in \pgf@xa
  \ifnum\tikz@spline@i=\tikz@spline@segments
    \advance\tikz@spline@i by-1
    \pgf@xa=1pt
  \else
    \pgf@xa=\tikz@time pt
    \pgf@xa=\tikz@spline@segments \pgf@xa
    \advance\pgf@xa by -\tikz@spline@i pt
  \fi
  % Determine the start and end of the segments
  \ifnum\tikz@spline@i=0
    \pgf@xb=\tikz@spline@starttime pt
  \else
    \pgf@xb=0pt
  \fi
  \c@pgf@countc=\tikz@spline@i
  \advance\c@pgf@countc by1
  \ifnum\c@pgf@countc=\tikz@spline@segments
    \pgf@yb=\tikz@spline@endtime pt
  \else
    \pgf@yb=1pt
  \fi
  % t' = (t - i/n)(yb - xb)/(1/n) = (t*n - i)(yb - xb) + xb
  % Put the scaling factor in \pgf@yb
  \advance\pgf@yb by-\pgf@xb
  % Compute scaled and shifted time
  \pgf@xa=\pgf@sys@tonumber{\pgf@yb}\pgf@xa
  \advance\pgf@xa by\pgf@xb
  \edef\tikz@spline@time{\pgf@sys@tonumber{\pgf@xa}}
  % Get the appropriate control points for the segment
  \tikz@spline@select@segment{\tikz@spline@i}%
  \tikz@spline@debug{%
    \string\tikz@time=\tikz@time\space -> \the\tikz@spline@i.\tikz@spline@time
  }%
  \pgftransformcurveattime{\tikz@spline@time}{\tikz@spline@zeroth}{\tikz@spline@first}{\tikz@spline@second}{\tikz@spline@third}%
}

\def\tikz@spline@coordinates{%
  \tikz@spline@get@K{\tikz@spline@i}%
  \tikz@spline@get@C{\tikz@spline@i}%
  \advance\tikz@spline@i by1
  \def\tikz@temp##1{\tikz@spline@name\space ##1-\the\tikz@spline@i}%
  \edef\tikz@marshal{%
    node[shape=coordinate](\tikz@temp{K})at(\the\pgf@x,\the\pgf@y){}%
    node[shape=coordinate](\tikz@temp{P})at(\the\pgf@xa,\the\pgf@ya){}%
    node[shape=coordinate](\tikz@temp{Q})at(\the\pgf@xb,\the\pgf@yb){}%
  }%
  \expandafter\endpgfextra\tikz@marshal
  \pgfextra
  \ifnum\tikz@spline@i=\tikz@spline@segments
    \iftikz@spline@closed
      \tikz@spline@get@K{0}%
    \else
      \tikz@spline@get@K{\tikz@spline@i}%
    \fi
    \advance\tikz@spline@i by1
    \edef\tikz@marshal{%
      node[shape=coordinate](\tikz@temp{K})at(\the\pgf@x,\the\pgf@y){}%
    }%
    \expandafter\endpgfextra\tikz@marshal
    \pgfextra
  \else
    \expandafter\tikz@spline@coordinates
  \fi
}
\endinput
% vim: set sw=2 sts=2 ts=8 et nospell:
